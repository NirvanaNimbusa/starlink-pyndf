\documentclass[11pt,nolof]{starlink}

% ? Specify used packages
% ? End of specify used packages

% ? Document identification
% Fixed part
%------------------------------------------------------------------------------
\stardoccategory    {Starlink User Note}
\stardocinitials    {SUN}
\stardocsource {sun\stardocnumber}

% Variable part
\stardocnumber      {39.2}
\stardocauthors     {R F Warren-Smith}
\stardocdate        {28 February 1995}
\stardoctitle       {PRIMDAT --- Processing of Primitive Numerical Data}
\stardocmanual {Programmer's Guide}
\stardocversion {v1.6}
\stardocabstract {%
The \name{PRIMDAT} package is a collection of Fortran functions
and subroutines providing support for primitive data processing.
}
% ? End of document identification
% -----------------------------------------------------------------------------
% ? Document specific \providecommand or \newenvironment commands.

\providecommand{\name}[1]{\mbox{#1}}
\providecommand{\fortvar}[1]{\mbox{\emph{#1}}}
\providecommand{\numcir}[1]{\mbox{\hspace{3ex}$\bigcirc$\hspace{-1.7ex}{\small #1}}}

% ? End of document specific commands
% -----------------------------------------------------------------------------
%  Title Page.
%  ===========
\begin{document}
\scfrontmatter

\section{INTRODUCTION}

This document describes version 1.0 of the \name{PRIMDAT} package, which
is a collection of Fortran functions and subroutines providing support for
``primitive data processing''.
Routines from this package may be used to perform arithmetic, mathematical
operations, type conversion and inter-comparison of any of the primitive
numerical data types supported by the Starlink Hierarchical Data System HDS
(SUN/92).

It provides:

\begin{itemize}

\item \textbf{Processing facilities which are not normally available.}\\
This package provides processing facilities for \textbf{all} the numerical
\name{HDS} data types.  In contrast, standard Fortran~77 only supports the
\name{\_INTEGER}, \name{\_REAL} and \name{\_DOUBLE} numerical types  and
even in \name{VAX} Fortran there are no operators for the unsigned types
\name{\_UWORD} and \name{\_UBYTE} (those for \name{\_BYTE} are also
somewhat restricted).

\item \textbf{A uniform interface.}\\
A consistent naming scheme is used which facilitates the production of \emph{generic} routines for processing data of any numerical type (\name{SUN/7}
describes the \name{GENERIC} compiler and contains further information on
this subject).

\item \textbf{Improved portability and efficiency.}\\
By using this package, applications can process non-standard data types
while making minimum explicit use of non-standard Fortran statements.
In addition, \name{PRIMDAT} routines can exploit machine-specific features
to enhance the efficiency of primitive data processing, so applications may
take advantage of these features without compromising their own portability.

\item \textbf{Facilities for processing \emph{bad} data.}\\
The \emph{bad values} defined by Starlink for flagging undefined data
may be processed and consistently propagated.
Recognition of these \emph{bad values} may easily be disabled when not
required in order to save processing time.

\item \textbf{Handling of numerical errors.}\\
Numerical errors (such as division by zero or numerical overflow) may be
handled automatically and converted into appropriate \emph{bad values} for
subsequent recognition.
Error codes are returned to allow such occurrences to be reported, or
recovery action taken, as desired.

\item \textbf{A set of constants.}\\
A set of machine-specific symbolic constants is provided to support the
processing facilities in this package and to simplify the task of
transporting software to other machines which may have different arithmetic
capabilities.

\end{itemize}

Note that only the \textbf{numerical} \name{HDS} data types (\name{\_BYTE},
\name{\_UBYTE}, \name{\_WORD}, \name{\_UWORD}, \name{\_INTEGER},
\name{\_REAL} \& \name{\_DOUBLE}) are supported at present.
There is currently no provision for the non-numerical \name{\_LOGICAL} and
\name{\_CHAR} types, nor for any other Fortran (or \name{VAX} Fortran) types
such as \name{COMPLEX}.
The \name{CHR} facility should be used for processing character data and for
converting between character and numerical data types.

Note added March 2004: This manual refers throughout to VAX floats.
The package was ported to Unix in 1991, and accordingly most of the
values now defined by it are those appropriate for IEEE floats
instead; the manual has not been updated to match.

\subsection{The \name{VAL}, \name{VEC} and \name{NUM} Facilities}

The routines in this package have names of the form
\mbox{$<$fac$>$\_$<$name$>$}, where \mbox{$<$fac$>$} is a three character
facility name and \mbox{$<$name$>$} specifies the operation which the
routine performs.

The routines are divided into three facilities as follows:

\begin{center}
\begin{tabular}{c|p{33em}}

\textbf{$<$fac$>$} & \textbf{Facility provides\ldots } \\
\hline
\\
VAL & Arithmetic, mathematical functions and type conversion on single
(scalar) \emph{values}.
Handling of numerical errors and \emph{bad value} propagation are
incorporated.\\
\\
VEC & Arithmetic, mathematical functions and type conversion on \emph{vectorised arrays}, allowing more efficient processing of large numbers of
data.
Handling of numerical errors and \emph{bad value} propagation are
incorporated.\\
\\
NUM & Lower level routines for arithmetic, mathematical functions, type
conversion and inter-comparison of single \emph{numbers}.
Operations are performed without protection against numerical errors and
without regard for \emph{bad} values.
\\
\end{tabular}
\end{center}

Note that a distinction is drawn between three classes of primitive data,
which differ in their interpretation and in the algorithms best suited to
processing them:

\begin{itemize}

\item \textbf{Values} (\name{VAL} facility) are scalar data which may take one
of the Starlink-defined \emph{bad values} (sometimes called ``magic''
values), whose presence signifies that the affected datum is undefined.
Numerical errors which occur while processing \emph{values} will result in
the generation of a \emph{bad value} as a result.
Subsequent recognition of \emph{bad values} on input may (or may not) occur,
depending on the logical setting of the \fortvar{BAD} routine argument
(section~\ref{section:bad}).

\item \textbf{Vectorised arrays} (\name{VEC} facility) are 1-dimensional arrays
of \emph{values} (or arrays treated as 1-dimensional) whose elements are
processed in the way described above.
The routines provided are optimised to perform processing in an efficient
manner when the size of the array is large.

\item \textbf{Numbers} (\name{NUM} facility) are always interpreted literally
as scalar numerical quantities (\emph{i.e. bad} values are not recognised on
input and are not explicitly generated on output).
Numerical errors which occur while processing \emph{numbers} will cause the
application to terminate (\emph{i.e.} crash).

\end{itemize}

Italics are used throughout this document to refer to \emph{values}, \emph{vectorised arrays} and \emph{numbers} when it is important to emphasise the
particular meaning (and the implied processing rules) described here.

\subsection{Propagation of \emph{Bad} Data}
\label{section:bad}

The handling of numerical errors by the \name{VAL\_} and \name{VEC\_}
routines\footnote{The \name{NUM\_} functions do not have a built-in error
handling capability and are not discussed here.}  depends on the generation,
recognition and propagation of \emph{bad values} in the data, and also on
the presence of an integer \fortvar{STATUS} argument carried by each
routine.

\emph{Bad values} are a set of special numerical constants (one for each data
type) which are defined by Starlink to be used for flagging ``bad'' or
``undefined'' data.
The philosophy and conventions surrounding their use are described in SGP/38.
The Starlink ``\emph{bad} constants'' which specify these values have
symbolic names\footnote{
These symbolic constants are defined in the include file \name{PRM\_PAR}.}
of the form:

\begin{terminalv}
VAL__BAD<T>
\end{terminalv}

where \name{$<$T$>$} is one of the data type codes in
Table~\ref{table:types} (also see section~\ref{section:badconstants}).
\emph{Bad values} may be set explicitly by assigning the value of one of
these constants to a variable.
They are also set by \name{VAL\_} and \name{VEC\_} routines when numerical
errors occur --- implying that the result of the attempted operation is
undefined.

\begin{table}
\begin{center}
\begin{tabular}{|l|rl|c|}
\hline
\textbf{Type Code $<$T$>$} &
\multicolumn{2}{|c|}{\textbf{HDS Type}} & \textbf{Fortran Type } \\
\hline
UB & \_UBYTE & (unsigned byte) & BYTE \\
B & \_BYTE & (byte) & BYTE \\
UW & \_UWORD & (unsigned word) & INTEGER$*$2 \\
W & \_WORD & (word) & INTEGER$*$2 \\
I & \_INTEGER & (integer) & INTEGER \\
R & \_REAL & (real) & REAL \\
D & \_DOUBLE & (double precision) & DOUBLE PRECISION \\
\hline
\end{tabular}
\caption{The data type codes used in constants and routine names, and the
\name{HDS} and \name{VAX} Fortran types to which they correspond.}
\label{table:types}
\end{center}
\end{table}

Recognition of \emph{bad values} involves comparing them for equality with
the appropriate ``\emph{bad} constant'', and may be used to control
subsequent processing.
Most frequently, this control takes the form of ``propagation'' --- \emph{i.e.} the outcome of any operation attempted on a \emph{bad} operand is
itself \emph{bad}; so that the \emph{bad value} is ``propagated'' to the
result.
After a sequence of such operations, the resulting \emph{bad values} make it
possible to identify all those results which have been rendered invalid by
errors occurring \emph{en route}.

The \name{VAL\_} and \name{VEC\_} routines have the ability to recognise
and propagate \emph{bad values} in this manner (the precise processing rules
are described in sections~\ref{section:val:error} \& \ref{section:vec:error}).
In some circumstances, however, appreciable amounts of processing time can
be saved if it is known in advance that \emph{bad values} are not present and
that their recognition is not necessary.
Also, it is occasionally necessary to disable \emph{bad value} recognition so
that the numerical values specified by the ``\emph{bad} constants'' may
themselves be processed as if they were valid data.

Because of this, recognition and propagation of \emph{bad values} by the
\name{VAL\_} and \name{VEC\_} routines is controlled by a logical value
\fortvar{BAD}, which appears at the start of each routine's argument list.
If \fortvar{BAD} is set \name{.TRUE.}, then \emph{bad} input arguments are
recognised and propagated to the result.
Conversely, if \fortvar{BAD} is set \name{.FALSE.}, then input argument
values are interpreted literally (there is no \emph{bad value} recognition),
although \emph{bad values} may still be generated on output if numerical
errors occur.
Most of the routines execute more efficiently if \fortvar{BAD} is
\name{.FALSE.} (Appendix~\ref{appendix:statistics}).

It is up to the calling routine to ensure that the \fortvar{BAD} argument
has an appropriate logical value when \name{VAL\_} and \name{VEC\_} routines
are invoked.
This value will often depend on whether numerical errors have occurred in
previous stages of processing, so each routine also carries a
\fortvar{STATUS} argument which may be used to detect the occurrence of such
errors.

\subsection{The \fortvar{STATUS} Argument}

Each \name{VAL\_} and \name{VEC\_} routine takes a standard
integer \fortvar{STATUS} variable as its final argument.
In normal use, this \fortvar{STATUS} variable should be set to
\name{SAI\_\_OK} when the routine is invoked, otherwise it will return
immediately without action.\footnote{
In the case of the \name{VAL\_} functions, a \emph{bad} function result will
be returned.}

On exit, the \fortvar{STATUS} value will still be set to \name{SAI\_\_OK}
unless a numerical error has occurred, in which case it will be set to an
appropriate error code (Appendix~\ref{appendix:errors}).
\name{PRIMDAT} error codes may be tested to detect when a particular
numerical error has occurred, and may also be used to report these errors
(using the \name{ADAM} \name{ERR\_} routines for instance).

Note that routines in this package do not report errors themselves.
This is partly an efficiency consideration.
In many circumstances, the automatic generation of a \emph{bad value} will
itself be an adequate response to a numerical error and, in such cases, the
\fortvar{STATUS} variable may simply be reset to \name{SAI\_\_OK} and
processing can continue.

\section{THE \name{VAL\_} FUNCTIONS}

\label{section:val}

The \name{VAL\_} functions are a set of Fortran functions which perform
arithmetic, mathematical operations and type conversion on single (scalar)
\emph{values} stored using any numerical data type.
These functions will handle numerical errors which occur during their
evaluation (such as division by zero or overflow) and can also recognise and
propagate the standard Starlink \emph{bad values}.

They are normally invoked by statements of the form:

\begin{terminalv}
RESULT = VAL_<name>( BAD, ARG, STATUS )
\end{terminalv}
or
\begin{terminalv}
RESULT = VAL_<name>( BAD, ARG1, ARG2, STATUS )
\end{terminalv}

where \name{$<$name$>$} specifies the operation to be performed and:

\begin{quote}
\begin{itemize}

\item \fortvar{BAD} is a logical value specifying whether \emph{bad}
input arguments are to be recognised;

\item \fortvar{ARG}, \fortvar{ARG1} and \fortvar{ARG2} are the input
arguments;

\item \fortvar{STATUS} is an integer error status variable.

\end{itemize}
\end{quote}

\subsection{Applicability \& Efficiency}

The \name{VAL\_} functions are designed primarily for ease of use and
robustness, with efficiency being a secondary consideration.
In fact, the
overhead involved in invoking many of these functions exceeds the time
required to actually perform the calculation.
Consequently, \name{VAL\_} functions should \textbf{\underline{not}} be used
for processing large arrays of data.
Rather, they are best used for relatively small numbers of calculations (up
to a few thousand), when they can enhance the robustness and flexibility of
applications by handling numerical errors and \emph{bad values} with the
minimum of programming effort.

The \name{VEC\_} routines (section~\ref{section:vec}) should normally be
used for processing larger arrays of data.

\subsection{\name{VAL\_} Error Handling}
\label{section:val:error}

The error handling strategy used by the \name{VAL\_} routines consists of
returning a \emph{bad} function result under the following circumstances:

\begin{enumerate}

\item If \fortvar{STATUS} is not set to \name{SAI\_\_OK} when the function
is invoked.

\item If the \fortvar{BAD} argument is set to \name{.TRUE.} and any of
the input arguments is \emph{bad}.

\item If any numerical error occurs during evaluation of the function --- in
this latter case an appropriate error code will also be returned via the
\fortvar{STATUS} argument.

\end{enumerate}

\subsection{\name{VAL\_} Arithmetic and Mathematical Functions}

Arithmetic and mathematical operations may be performed on scalar \emph{values} stored using any numerical data type by invoking appropriate
\name{VAL\_} functions, using statements such as:

\begin{terminalv}
RESULT = VAL_<FUNC><T>( BAD, ARG, STATUS )
\end{terminalv}
or
\begin{terminalv}
RESULT = VAL_<FUNC><T>( BAD, ARG1, ARG2, STATUS )
\end{terminalv}

Here, the \name{VAL\_} function takes an argument (or arguments) of type
\name{$<$T$>$} and returns a result which is also of type \name{$<$T$>$},
having performed the arithmetic or mathematical operation specified by
\name{$<$FUNC$>$}.
The data type code \name{$<$T$>$} may be any of those specified in
Table~\ref{table:types} and \name{$<$FUNC$>$} may be any of the function
codes specified in Tables~\ref{table:functions} \&
\ref{table:floatfunctions}.
The number of input arguments (1 or 2) appropriate to each function is also
indicated in these latter Tables.

\begin{table}
\begin{center}
\begin{tabular}{|c|c|l|}
\hline
\begin{tabular}{c}
\textbf{Function Code } \\ \textbf{$<$FUNC$>$ }
\end{tabular} &
\begin{tabular}{c}
\textbf{Number of} \\
\textbf{Arguments}
\end{tabular} &
\textbf{Operation performed} \\
\hline
ADD & 2 & addition: ARG1 $+$ ARG2 \\
SUB & 2 & subtraction: ARG1 $-$ ARG2 \\
MUL & 2 & multiplication: ARG1 $*$ ARG2 \\
DIV & 2 & *(floating) division: ARG1 $/$ ARG2 \\
IDV & 2 & **(integer) division: ARG1 $/$ ARG2 \\
PWR & 2 & raise to power: ARG1 $**$ ARG2 \\
NEG & 1 & negate (change sign): $-$ARG \\
SQRT & 1 & square root: $\sqrt{\mbox{ARG}}$ \\
LOG & 1 & natural logarithm: $\ln (\mbox{ARG})$ \\
LG10 & 1 & common logarithm: $\log _{10}(\mbox{ARG})$ \\
EXP & 1 & exponential: $\exp (\mbox{ARG})$ \\
ABS & 1 & absolute (positive) value: $\left| \mbox{ARG} \right|$ \\
NINT & 1 & nearest integer value to ARG \\
INT & 1 & Fortran AINT (truncation to integer) function \\
MAX & 2 & maximum: $\mbox{max}(\mbox{ARG1},\mbox{ARG2})$ \\
MIN & 2 & minimum: $\mbox{min}(\mbox{ARG1},\mbox{ARG2})$ \\
DIM & 2 & Fortran DIM (positive difference) function \\
MOD & 2 & Fortran MOD (remainder) function \\
SIGN & 2 & Fortran SIGN (transfer of sign) function \\
\hline
\multicolumn{1}{|r}{\footnotesize Notes:} &
\multicolumn{2}{l|}{\footnotesize *Equivalent to
NINT(REAL(ARG1)$/$REAL(ARG2)) for non-floating quantities} \\
\multicolumn{1}{|l}{} &
\multicolumn{2}{l|}{\footnotesize **Equivalent to AINT(ARG1$/$ARG2) for
floating quantities} \\
\hline
\end{tabular}
\caption{Function codes used in routine names and the operations to which they
correspond.
The functions shown here are implemented for \textbf{all} the numerical data
types (type codes UB, B, UW, W, I, R \& D).}
\label{table:functions}
\end{center}
\end{table}

Thus, for example, the function \name{VAL\_ADDUW} adds two unsigned word
arguments to produce an unsigned word result, while \name{VAL\_SQRTD}
evaluates the square root of a double precision argument, returning a double
precision result.

\subsection{\name{VAL\_} Type Conversion Functions}

Conversion of scalar \emph{values} between numerical data types may be
performed by invoking the appropriate \name{VAL\_} type conversion functions,
using statements such as:

\begin{terminalv}
RESULT = VAL_<T1>TO<T2>( BAD, ARG, STATUS )
\end{terminalv}

Here, the \name{VAL\_} function takes an argument of type \name{$<$T$_{1}>$}
and returns a result of type \name{$<$T$_{2}>$}, having performed type
conversion.
The data type codes \name{$<$T$_{1}>$} and \name{$<$T$_{2}>$} may be any of
those specified in Table~\ref{table:types}.

Thus, for example, the function \name{VAL\_BTOUW} converts a byte argument
to an unsigned word result, while \name{VAL\_DTOI} converts from double
precision to integer.

Note that conversions from floating point types (\name{\_REAL} or
\name{\_DOUBLE}) to non-floating point types result in rounding of the data
(not truncation as would happen with a Fortran assignment statement).

\begin{table}
\begin{center}
\begin{tabular}{|c|c|l|}
\hline
\begin{tabular}{c}
\textbf{Function Code } \\ \textbf{$<$FUNC$>$ }
\end{tabular} &
\begin{tabular}{c}
\textbf{Number of} \\
\textbf{Arguments}
\end{tabular} &
\textbf{Operation performed} \\
\hline
SIN & 1 & *sine function: $\sin (\mbox{ARG})$ \\
SIND & 1 & **sine function: $\sin (\mbox{ARG})$ \\
COS & 1 & *cosine function: $\cos (\mbox{ARG})$ \\
COSD & 1 & **cosine function: $\cos (\mbox{ARG})$ \\
TAN & 1 & *tangent function: $\tan (\mbox{ARG})$ \\
TAND & 1 & **tangent function: $\tan (\mbox{ARG})$ \\
ASIN & 1 & *inverse sine function: $\sin ^{-1}(\mbox{ARG})$ \\
ASND & 1 & **inverse sine function: $\sin ^{-1}(\mbox{ARG})$ \\
ACOS & 1 & *inverse cosine function: $\cos ^{-1}(\mbox{ARG})$ \\
ACSD & 1 & **inverse cosine function: $\cos ^{-1}(\mbox{ARG})$ \\
ATAN & 1 & *inverse tangent function: $\tan ^{-1}(\mbox{ARG})$ \\
ATND & 1 & *inverse tangent function: $\tan ^{-1}(\mbox{ARG})$ \\
ATN2 & 2 & *Fortran ATAN2 (inverse tangent) function \\
AT2D & 2 & **VAX Fortran ATAN2D (inverse tangent) function \\
SINH & 1 & hyperbolic sine function: $\sinh (\mbox{ARG})$ \\
COSH & 1 & hyperbolic cosine function: $\cosh (\mbox{ARG})$ \\
TANH & 1 & hyperbolic tangent function: $\tanh (\mbox{ARG})$ \\
\hline
\multicolumn{1}{|r}{\footnotesize Notes:} &
\multicolumn{2}{l|}{\footnotesize *Argument(s)/result in radians} \\
\multicolumn{1}{|l}{} &
\multicolumn{2}{l|}{\footnotesize **Argument(s)/result in degrees} \\
\hline
\end{tabular}
\caption{Function codes used in routine names and the operations to which they
correspond.
The functions shown here are only implemented for the floating point data types
(type codes R \& D).}
\label{table:floatfunctions}
\end{center}
\end{table}

\subsection{Declaring the \name{VAL\_} Functions}

Since the \name{VAL\_} functions return a result via the routine name,
appropriate declarations of their Fortran type must be made before they are
used.
Thus, for instance, an application which used the routines \name{VAL\_ADDUB}
and \name{VAL\_DTOI} would require the declarations:

\begin{terminalv}
BYTE VAL_ADDUB
INTEGER VAL_DTOI
\end{terminalv}

A routine suitable for processing by the \name{GENERIC} compiler might
require equivalent generic declarations:

\begin{terminalv}
<TYPE> VAL_ADD<T>
<TYPE> VAL_DTO<T>
\end{terminalv}

\section{THE \name{VEC\_} ROUTINES}

\label{section:vec}

The \name{VEC\_} routines are a set of subroutines which perform arithmetic,
mathematical operations and type conversion on \emph{vectorised arrays} of
data stored using any numerical type.
They incorporate handling of numerical errors and propagation of the
standard \emph{bad values} in a way that is fully compatible with the
\name{VAL\_} functions.

\name{VEC\_} routines are invoked by statements of the form:

\begin{terminalv}
CALL VEC_<name>( BAD, N, ARG, RESULT, IERR, NERR, STATUS )
\end{terminalv}
or
\begin{terminalv}
CALL VEC_<name>( BAD, N, ARG1, ARG2, RESULT, IERR, NERR, STATUS )
\end{terminalv}

where \name{$<$name$>$} specifies the operation to be performed and:

\begin{quote}
\begin{itemize}

\item \fortvar{BAD} is a logical value specifying whether \emph{bad}
input arguments are to be recognised;

\item \fortvar{N} is an integer value specifying the number of array
elements to process;

\item \fortvar{ARG}, \fortvar{ARG1} and \fortvar{ARG2} are the input
arguments (each is an \fortvar{N}-element \emph{vectorised array});

\item \fortvar{RESULT} is an \fortvar{N}-element \emph{vectorised} output
array;

\item \fortvar{IERR} is an integer output argument which identifies the
first array element to generate a numerical error;

\item \fortvar{NERR} is an integer output argument which returns a count of the
number of numerical errors which occur;

\item \fortvar{STATUS} is an integer error status variable.

\end{itemize}
\end{quote}

\subsection{Applicability \& Efficiency}

The \name{VEC\_} routines are designed for optimum efficiency when
processing large arrays of data and are consequently \textbf{\underline{far}}
more efficient at performing this task than the corresponding \name{VAL\_}
functions.
When possible, \name{VEC\_} routines should always be used in preference to
\name{VAL\_} functions unless the number of calculations to be performed is
small (typically 1000 or less).

The main disadvantage in using \name{VEC\_} routines is likely to be the
additional programming effort, because workspace arrays will probably be
required and may have to be acquired dynamically --- unless only a single
operation is being performed.
Also, certain operations cannot be performed using \name{VEC\_} routines
(summing the pixels in an image for example) and, in such cases,
routines from the \name{NUM} facility (section~\ref{section:num}) may have
to be used instead.
Note, however, that if \name{NUM\_} routines are used, then security against
numerical errors is more difficult to achieve and (especially in the case of
arithmetic operations) \name{VAX}-specific programming techniques may be
required.
The resulting applications will not then be portable.

\subsection{\name{VEC\_} Error Handling}
\label{section:vec:error}

The error handling strategy used by the \name{VEC\_} routines is as follows:

\begin{enumerate}

\item If the \fortvar{STATUS} argument is not set to \name{SAI\_\_OK} on
entry, then the routine returns immediately without action ---
the contents of the \fortvar{RESULT} array remain undefined in this case.

\item If the \fortvar{BAD} argument is set to \name{.TRUE.} and any
element of an input argument array is \emph{bad}, then a \emph{bad value} is
placed in the corresponding element of the \fortvar{RESULT} array.

\item A \emph{bad value} is also placed in the corresponding element of the
\fortvar{RESULT} array if a numerical error occurs when processing any
element of an input argument array --- in this instance an appropriate
error code will also be returned via the \fortvar{STATUS} argument.

\end{enumerate}

On exit, the \fortvar{IERR} argument will contain the index of the first
array element which generated a numerical error (or zero if there were no
errors) and \fortvar{NERR} will contain a count of the number of
numerical errors which occurred.
If more than one error occurred, then the \fortvar{STATUS} value returned will
be appropriate to the \textbf{first} of these.

\subsection{\name{VEC\_} Arithmetic and Mathematical Routines}

Arithmetic and mathematical operations may be performed on \emph{vectorised
arrays} of data of any numerical type by invoking appropriate \name{VEC\_}
routines, using statements such as:

\begin{terminalv}
CALL VEC_<FUNC><T> ( BAD, N, ARG, RESULT, IERR, NERR, STATUS )
\end{terminalv}
or
\begin{terminalv}
CALL VEC_<FUNC><T> ( BAD, N, ARG1, ARG2, RESULT, IERR, NERR, STATUS )
\end{terminalv}

Here, the \name{VEC\_} routine takes a \emph{vectorised} input argument array
(or arrays) with \fortvar{N} elements of type \name{$<$T$>$} and returns a \emph{vectorised} result array which is also of type \name{$<$T$>$}, having performed
the arithmetic or mathematical operation specified by \name{$<$FUNC$>$} on
each array element.
The data type code \name{$<$T$>$} may be any of those specified in
Table~\ref{table:types} and \name{$<$FUNC$>$} may be any of the function
codes specified in Tables~\ref{table:functions} \&
\ref{table:floatfunctions}.
The number of input argument arrays (1 or 2) appropriate to each function is
also indicated in these latter Tables.

Thus, for example, the subroutine \name{VEC\_ADDUW} adds two \emph{vectorised}
unsigned word arrays to produce an array of unsigned word results, while
\name{VEC\_SQRTD} evaluates the square root of a single \emph{vectorised
array} of double precision \emph{values}, returning an array of double
precision results.

\subsection{\name{VEC\_} Type Conversion Routines}

Conversion of \emph{vectorised arrays} between numerical data types may be
performed by invoking the appropriate \name{VEC\_} type conversion routines,
using statements such as:

\begin{terminalv}
CALL VEC_<T1>TO<T2> ( BAD, N, ARG, RESULT, IERR, NERR, STATUS )
\end{terminalv}

Here, the \name{VEC\_} routine takes a \emph{vectorised} input argument array
with \fortvar{N} elements of type \name{$<$T$_{1}>$} and returns a \emph{vectorised} result array of type \name{$<$T$_{2}>$}, having performed type
conversion on each array element.
The data type codes \name{$<$T$_{1}>$} and \name{$<$T$_{2}>$} may be any of
those specified in Table~\ref{table:types}.

Thus, for example, the subroutine \name{VEC\_BTOUW} converts a \emph{vectorised array} of byte data into unsigned word data, while
\name{VEC\_DTOI} converts a \emph{vectorised array} from double precision to
integer.

Note that conversions from floating point types (\name{\_REAL} or
\name{\_DOUBLE}) to non-floating point types result in rounding of the data
(not truncation as would happen with a Fortran assignment statement).

\section{THE NUM\_ FUNCTIONS}

\label{section:num}

The \name{NUM\_} functions perform arithmetic, mathematical operations, type
conversion and inter-comparison on \emph{numbers}, operating at a rather
lower level than the \name{VAL\_} or \name{VEC\_} routines --- in fact many
of these latter routines are currently implemented using the \name{NUM}
facility.

At present, all the \name{NUM\_} routines are implemented as Fortran
statement functions whose definitions must be incorporated into an
application with appropriate \name{INCLUDE} statements
(section~\ref{section:include}).

They are normally invoked by statements of the form:

\begin{terminalv}
RESULT = NUM_<name>( ARG )
\end{terminalv}
or
\begin{terminalv}
RESULT = NUM_<name>( ARG1, ARG2 )
\end{terminalv}

where \name{$<$name$>$} specifies the operation to be performed and
\fortvar{ARG}, \fortvar{ARG1} and \fortvar{ARG2} are input arguments.

\subsection{Applicability \& Efficiency}

The \name{NUM\_} functions are the most flexible routines in this package.
Being implemented as statement functions, they will normally be expanded
``in-line'' by the \name{VAX}~Fortran compiler and may therefore be used in
any situation without incurring the overhead of an external function call.
The expanded function definition can subsequently take part in the
compiler's optimisation process and will therefore result in code which
executes very efficiently (although some of the machine-specific features
used by \name{VEC\_} routines cannot be accessed in this way).

However, the \name{NUM\_} functions are also the least robust routines in
this package.
This is because they incorporate no protection against numerical errors
(such as division by zero or overflow), neither do they generate or
recognise \emph{bad} data --- which they will always treat as valid.
If a numerical error occurs when a \name{NUM\_} function is invoked, the
application will immediately terminate (\emph{i.e.} crash).

The robustness of \name{NUM\_} functions may be improved by incorporating
explicit checks on the \emph{numbers} being processed and by ensuring that
the functions are not invoked if their arguments may be undefined (under
error conditions, for instance).
Such checks will always degrade the execution efficiency, however, and the
equivalent \name{VEC\_} routine (section~\ref{section:vec}) is then likely
to be at least as efficient.
Consequently, use of the \name{NUM\_} functions is generally appropriate
only when:

\begin{itemize}

\item It is known (from the nature of the data) that numerical errors cannot
occur; or

\item Efficiency considerations completely outweigh the need for robustness;
or

\item The problem is such that no alternative routine from this package
is applicable.

\end{itemize}

For applications where \name{VAX}-specific code is acceptable, the
\name{NUM} facility also includes a simple \name{VAX} condition handler
which can be used to recover from numerical errors
(section~\ref{section:num:error}).
This may be used with the \name{NUM\_} functions (and Fortran arithmetic
expressions) to combine robustness with efficiency and flexibility in cases
where other routines from this package are not suitable.
However, the programming effort required is relatively high and the
resulting applications will not be portable.

\subsection{\name{NUM\_} Arithmetic and Mathematical Functions}

Arithmetic and mathematical operations may be performed on \emph{numbers}
stored using any numerical data type by invoking appropriate \name{NUM\_}
functions, using statements such as:

\begin{terminalv}
RESULT = NUM_<FUNC><T>( ARG )
\end{terminalv}
or
\begin{terminalv}
RESULT = NUM_<FUNC><T>( ARG1, ARG2 )
\end{terminalv}

Here, the \name{NUM\_} function takes an argument (or arguments) of type
\name{$<$T$>$} and returns a result which is also of type \name{$<$T$>$},
having performed the arithmetic or mathematical operation specified by
\name{$<$FUNC$>$}.
The data type code \name{$<$T$>$} may be any of those specified in
Table~\ref{table:types} and \name{$<$FUNC$>$} may be any of the function
codes specified in Tables~\ref{table:functions} \&
\ref{table:floatfunctions}.
The number of function arguments (1 or 2) appropriate to each function is
also indicated in these latter Tables.

Thus, for example, the function \name{NUM\_ADDUW} adds two unsigned word
arguments to produce an unsigned word result, while \name{NUM\_SQRTD}
evaluates the square root of a double precision argument, returning a double
precision result.

\subsection{NUM\_ Type Conversion Functions}

Conversion of \emph{numbers} between numerical data types may be performed by
invoking the appropriate \name{NUM\_} type conversion functions, using
statements such as:

\begin{terminalv}
RESULT = NUM_<T1>TO<T2>( ARG )
\end{terminalv}

Here, the \name{NUM\_} function takes an argument of type \name{$<$T$_{1}>$}
and returns a result of type \name{$<$T$_{2}>$}, having performed type
conversion.
The data type codes \name{$<$T$_{1}>$} and \name{$<$T$_{2}>$} may be any of
those specified in Table~\ref{table:types}.

Thus, for example, the function \name{NUM\_BTOUW} converts a byte argument
to an unsigned word result, while \name{NUM\_DTOI} converts from double
precision to integer.

Note that conversions from floating point types (\name{\_REAL} or
\name{\_DOUBLE}) to non-floating point types result in rounding of the data
(not truncation as would happen with a Fortran assignment statement).

\subsection{NUM\_ Inter-comparison Functions}
\label{section:logicals}

Inter-comparison of \emph{numbers} stored using any numerical data type, to
test for equality or inequality relations, may be performed by invoking the
appropriate \name{NUM\_} inter-comparison functions, using statements such as:

\begin{terminalv}
RESULT = NUM_<ROPER><T>( ARG1, ARG2 )
\end{terminalv}

Here, the \name{NUM\_} function takes two arguments of type \name{$<$T$>$}
and returns a logical result, having compared its arguments according to the
relational operation specified by \name{$<$ROPER$>$}.
The data type code \name{$<$T$>$} may be any of those in
Table~\ref{table:types} and \name{$<$ROPER$>$} may be any of the relational
operation codes specified in Table~\ref{table:logicals}.

\begin{table}
\begin{center}
\begin{tabular}{|c|c|}
\hline
\begin{tabular}{c}
\textbf{Relational Operation Code } \\ \textbf{$<$ROPER$>$ }
\end{tabular}
& \textbf{Inter-comparison} \\
\hline
EQ & ARG1 .EQ. ARG2 \\
NE & ARG1 .NE. ARG2 \\
GT & ARG1 .GT. ARG2 \\
GE & ARG1 .GE. ARG2 \\
LT & ARG1 .LT. ARG2 \\
LE & ARG1 .LE. ARG2 \\
\hline
\end{tabular}
\caption{The relational operation codes used in the names of \name{NUM\_}
functions and the inter-comparisons to which they correspond.}
\label{table:logicals}
\end{center}
\end{table}

Thus, for example, the logical function \name{NUM\_GTUB} returns a
\name{.TRUE.} result if the unsigned byte \emph{number} \name{ARG1} is
greater than the unsigned byte \emph{number} \name{ARG2}, while
\name{NUM\_LED} returns \name{.TRUE.} if the double precision \emph{number}
\name{ARG1} is less than or equal to the double precision \emph{number}
\name{ARG2}.

Note that equivalent inter-comparison functions are not available in the
\name{VAL} or \name{VEC} facilities.

\subsection{Declaring and Defining the NUM\_ Functions}
\label{section:include}

The \name{NUM} facility is implemented as a set of Fortran statement
functions.  To use them, an application must:

\begin{enumerate}

\item Declare the function names and dummy arguments, specifying their
Fortran types;

\item Define the functions.

\end{enumerate}

Two files are provided which contain Fortran statements to perform these
tasks and they may be included in an application with the statements:

\begin{terminalv}
INCLUDE 'NUM_DEC'
INCLUDE 'NUM_DEF'
\end{terminalv}

which should appear (in this order) immediately after any local Fortran
variables have been declared and before the executable code begins.
These statements will declare and define the entire set of \name{NUM\_}
functions.

In almost all cases, however, compilation time can be considerably reduced
by defining only a sub-set of these functions.
Files are therefore provided which allow the function definitions to be
selected according to the data type of their arguments.
When using this option, the following two statements should first be used to
declare and define the type conversion functions and their arguments (since
these are used by all the other \name{NUM\_} functions they must always be
present):

\begin{terminalv}
INCLUDE 'NUM_DEC_CVT'
INCLUDE 'NUM_DEF_CVT'
\end{terminalv}

If the application only uses the \name{NUM\_} functions to perform type
conversion, then this is all that is required.
However, to additionally declare and define the \name{NUM\_} functions which
process a particular argument data type, the statements:

\begin{terminalv}
INCLUDE 'NUM_DEC_<T>'
INCLUDE 'NUM_DEF_<T>'
\end{terminalv}

should then be used (here, the argument type code \name{$<$T$>$} is one of
those specified in Table~\ref{table:types}).
Thus, to give a complete example, the \name{NUM\_} functions for processing
unsigned word data would be declared and defined using the following four
statements:

\begin{terminalv}
INCLUDE 'NUM_DEC_CVT'
INCLUDE 'NUM_DEC_UW'
INCLUDE 'NUM_DEF_CVT'
INCLUDE 'NUM_DEF_UW'
\end{terminalv}

Note that to comply with Fortran restrictions on statement order, all the
type declaration statements (\name{NUM\_DEC\_} files) have to appear \textbf{before} the function definition statements (\name{NUM\_DEF\_} files).

\subsection{\name{NUM} Error Handling}
\label{section:num:error}

Because the \name{NUM\_} functions do not have any built-in error handling
capability, the \name{NUM} facility provides a simple \name{VAX} condition
handler which may be used to explicitly implement error handling, both for
the \name{NUM\_} functions and straightforward Fortran arithmetic
expressions.

\emph{Note that using this capability involves calls to \name{VAX} Run
Time Library (\name{RTL}) routines and will therefore result in applications
which are not portable.
Such applications will continue to operate, however, if the \name{RTL} calls
are simply removed, although error handling will not then occur.
}

The technique is best explained by an example:

\begin{small}
\begin{terminalv}
      SUBROUTINE EXAMPL( N, DATA )

*   Declare constants and variables.
      INCLUDE 'SAE_PAR'
      INCLUDE 'PRM_PAR'
      INTEGER N, I
      REAL DATA( N )

*   Declare the condition handler.
      INTEGER NUM_TRAP                            [1]
      EXTERNAL NUM_TRAP

*   Include the NUM_CMN common block definition, which defines
*   the variable NUM_ERROR.
      INCLUDE 'NUM_CMN'                           [2]

*   Establish the condition handler.
      CALL LIB$ESTABLISH( NUM_TRAP )              [3]

*   Initialise.
      NUM_ERROR = SAI__OK                         [4]

*   Perform the calculations.
      DO 1 I = 1, N
         DATA( I ) = 1.0 / ( DATA( I ) ** 2 )     [5]

*   Check if an error occurred.
         IF( NUM_ERROR .NE. SAI__OK ) THEN        [6]

*   If so, reset the NUM_ERROR flag and define the result.
            NUM_ERROR = SAI__OK                   [7]
            DATA( I ) = VAL__BADR
         ENDIF
    1 CONTINUE

*   Remove the condition handler and exit the routine.
      CALL LIB$REVERT                             [8]
      END
\end{terminalv}
\end{small}

Programming notes:

\begin{enumerate}

\item The integer function \name{NUM\_TRAP} is provided as a standard
condition handler which traps numerical errors.
This routine must be declared in an EXTERNAL statement.

\item The common block \name{NUM\_CMN} is defined using an \name{INCLUDE}
statement.
It contains a single integer variable called \name{NUM\_ERROR} which
communicates with the condition handler.

\item The \name{VAX} \name{RTL} routine \name{LIB\$ESTABLISH} is called to
establish the condition handler.
From this point on, numerical errors will not cause the application to
crash, but will be intercepted by the \name{NUM\_TRAP} routine.
This behaviour is local to the routine from which \name{LIB\$ESTABLISH} is
called.
The behaviour of \name{VAL\_} and \name{VEC\_} routines and the handling of
non-numerical errors is not affected.

\item The common block variable \name{NUM\_ERROR} is initialised to
\name{SAI\_\_OK}.
This is important because numerical errors will not be detected otherwise.

\item Calculations are performed which may potentially fail.  These may
include calls to \name{NUM\_} functions and need not be confined to a single
expression or statement.
In the above example floating point overflow or division by zero could
occur, according to the contents of the \fortvar{DATA} array.

\item  After a calculation, the status variable \name{NUM\_ERROR} is tested.
If it is not \name{SAI\_\_OK}, then a numerical error has occurred.
The value of \name{NUM\_ERROR} may be used as a status code for reporting
the error if required.

\item  If an error has occurred, suitable action is taken.
This must include resetting the numerical error status \name{NUM\_ERROR} to
\name{SAI\_\_OK} and defining the result of any calculations which failed.
This will usually be done using a flag for later identification (normally
one of the Starlink \emph{bad values}).

\item Finally, the condition handler is removed by calling the \name{RTL}
routine \name{LIB\$REVERT}.  This is strictly only necessary if further
calculations which do not require error handling are to be performed in the
same routine.

\end{enumerate}

It is important to note that when a calculation fails, more than
one numerical error may occur.
This is because the (erroneous) result of the first operation to fail can be
re-used in subsequent parts of an arithmetic expression and may precipitate
further errors.
To ensure that \name{NUM\_ERROR} indicates the true (first) cause of an error
in such cases, this variable behaves as a latch; once set, it will not again
change in response to an error until it has first been reset to
\name{SAI\_\_OK}.

Note also, that the result of a calculation which fails (or may have
failed) must \textbf{\underline{not}} be used \textbf{under any circumstances}.
It may either contain an erroneous result or an invalid floating point
number (a reserved operand) which will cause any application which processes
it to crash.
Such results \textbf{must} immediately be replaced with a well-defined value.

\section{CONSTANTS}

\label{section:constants}

This section describes a set of constants which support the processing
facilities in this package by specifying machine-specific quantities
associated with each data type.
These constants are identified by symbolic names which are defined (using
Fortran \name{PARAMETER} statements) in the file \name{PRM\_PAR}.
The definitions may be incorporated into an application with the statement:

\begin{terminalv}
INCLUDE 'PRM_PAR'
\end{terminalv}

Here, as elsewhere in this package, a distinction is drawn between \emph{values} (which may be flagged as \emph{bad}) and \emph{numbers} (which are
always interpreted literally).
The constants accordingly have names with one of the two forms:

\begin{terminalv}
VAL__<const><T>
\end{terminalv}
or
\begin{terminalv}
NUM__<const><T>
\end{terminalv}

where \name{$<$T$>$} is one of the type codes in Table~\ref{table:types} and
\name{$<$const$>$} specifies the quantity which the constant represents.
In most cases, the two forms of the constant are necessarily identical and
only the \name{VAL\_\_} form is provided.

The constants are also available in the C include file \name{prm\_par.h}.

Note that many of these constants may need to change if transported to
another machine.\footnote{
The constants given here are appropriate to \name{VAX} machines using F- and
D-type floating point number representations.
Only a simple description of floating point arithmetic is given.
Applications requiring a more complete description should use routines
from the \name{NAG} library, with which the values given here are compatible.}
When writing software which is intended to be portable you should therefore:

\begin{itemize}

\item Use the constants provided here, rather than defining your own;

\item Always refer to constants using symbolic names (\textbf{never} explicit
numerical values);

\item Never make implicit assumptions about the values these constants might
take.

\end{itemize}

A summary of the constants available is given in
Table~\ref{table:constants}.
The following sections describe them in more detail.

\begin{table}
\begin{center}
\begin{tabular}{|l|c|p{25em}|}
\hline
\textbf{Constant} & \textbf{Type} & \textbf{Quantity represented} \\
\hline
VAL\_\_BAD$<$T$>$ & *$<$T$>$ & \emph{Bad value}, used for flagging undefined
data.
\\
VAL\_\_EPS$<$T$>$ & $<$T$>$ & Machine precision.
\\
VAL\_\_MAX$<$T$>$ & $<$T$>$ & Maximum (most positive) non-\emph{bad value}.
\\
NUM\_\_MAX$<$T$>$ & $<$T$>$ & Maximum (most positive) \emph{number}.
\\
VAL\_\_MIN$<$T$>$ & $<$T$>$ & Minimum (most negative) non-\emph{bad value}.
\\
NUM\_\_MIN$<$T$>$ & $<$T$>$ & Minimum (most negative) \emph{number}.
\\
VAL\_\_NB$<$T$>$ & \_INTEGER & Number of basic machine units (bytes) used
by a value.
\\
VAL\_\_SML$<$T$>$ & $<$T$>$ & Smallest positive (non-zero) value.
\\
VAL\_\_SZ$<$T$>$ & \_INTEGER & Number of characters required to format a
value as a decimal string.
\\
\hline
\multicolumn{1}{|c}{} &
\multicolumn{2}{l|}{\footnotesize *Type=$<$T$>$ indicates that the
constant's data type matches the type code $<$T$>$.} \\
\hline
\end{tabular}
\caption{A summary of the constants available, showing their data types and
the quantities they represent.}
\label{table:constants}
\end{center}
\end{table}

\subsection{Bad Data Values}
\label{section:badconstants}

Constants with names of the form \name{VAL\_\_BAD$<$T$>$} represent the
Starlink-defined \emph{bad} (or ``magic'') \emph{values} to be used for
flagging \emph{bad} or undefined data.
The use of these constants is discussed in SGP/38, which also defines the
values to be used on \name{VAX} machines.
These are:

\begin{quote}
\begin{center}
\begin{tabular}{|l|l|r|r|}
\hline
\multicolumn{1}{|c|}{\textbf{Constant}} &
\multicolumn{1}{c|}{\textbf{Type}} &
\multicolumn{1}{c|}{\textbf{Value}} &
\multicolumn{1}{c|}{\textbf{Hexadecimal Pattern}} \\
\hline
VAL\_\_BADUB & \_UBYTE & 255 & FF \\
VAL\_\_BADB & \_BYTE & $-$128 & 80 \\
VAL\_\_BADUW & \_UWORD & 65535 & FFFF \\
VAL\_\_BADW & \_WORD & $-$32768 & 8000 \\
VAL\_\_BADI & \_INTEGER & $-$2147483648 & 80000000 \\
VAL\_\_BADR & \_REAL & $-$1.7014117E$+$38 & FFFFFFFF \\
VAL\_\_BADD & \_DOUBLE & $-$1.701411834604923D$+$38 & FFFFFFFFFFFFFFFF \\
\hline
\end{tabular}
\end{center}
\end{quote}

Note that the operations permitted on \emph{bad values} are restricted to:

\begin{itemize}

\item Assigning the \emph{value} of a ``\emph{bad} constant'' to a variable;

\item Testing a variable for equality with a ``\emph{bad} constant'';

\item Assigning a non-\emph{bad value} to a variable to replace one which was
previously \emph{bad}.

\end{itemize}

Software which aims to be portable may assume that the \emph{bad values}
described by these constants will always lie at the extreme end of the
\emph{number} range which can be represented by each data type.
However, no assumptions should be made about \textbf{which} end of the range
(upper or lower) this will be --- an explicit test should be made if it is
necessary to determine this.

\subsection{Machine Precision}

Constants with names of the form \name{VAL\_\_EPS$<$T$>$} represent the
machine precision when performing calculations using each data type.
The machine precision $\epsilon$ is the smallest positive value such that
\mbox{($1+\epsilon)$} can be represented on the machine and is distinguishable
from 1 using the data type in question.
The data types of these constants match the quantities they describe, as
follows:

\begin{quote}
\begin{center}
\begin{tabular}{|l|l|r|r|}
\hline
\multicolumn{1}{|c|}{\textbf{Constant}} &
\multicolumn{1}{c|}{\textbf{Type}} &
\multicolumn{1}{c|}{\textbf{Value}} &
\multicolumn{1}{c|}{\textbf{Hexadecimal Pattern}} \\
\hline
VAL\_\_EPSUB & \_UBYTE & 1 & 01 \\
VAL\_\_EPSB & \_BYTE & 1 & 01 \\
VAL\_\_EPSUW & \_UWORD & 1 & 0001 \\
VAL\_\_EPSW & \_WORD & 1 & 0001 \\
VAL\_\_EPSI & \_INTEGER & 1 & 00000001 \\
VAL\_\_EPSR & \_REAL & 1.1920929E$-$7 & 00003500 \\
VAL\_\_EPSD & \_DOUBLE & 2.7755575615628914E$-$17 & 0000000000002500 \\
\hline
\end{tabular}
\end{center}
\end{quote}

\subsection{Maximum Data Values}

Constants with names of the form \name{VAL\_\_MAX$<$T$>$} represent the
maximum (most positive) non-\emph{bad values} which can be represented
on the machine.
The data types of these constants match the quantities they describe, as
follows:

\begin{quote}
\begin{center}
\begin{tabular}{|l|l|r|r|}
\hline
\multicolumn{1}{|c|}{\textbf{Constant}} &
\multicolumn{1}{c|}{\textbf{Type}} &
\multicolumn{1}{c|}{\textbf{Value}} &
\multicolumn{1}{c|}{\textbf{Hexadecimal Pattern}} \\
\hline
VAL\_\_MAXUB & \_UBYTE & 254 & FE \\
VAL\_\_MAXB & \_BYTE & 127 & 7F \\
VAL\_\_MAXUW & \_UWORD & 65534 & FFFE \\
VAL\_\_MAXW & \_WORD & 32767 & 7FFF \\
VAL\_\_MAXI & \_INTEGER & 2147483647 & 7FFFFFFF \\
VAL\_\_MAXR & \_REAL & 1.7014117E$+$38 & FFFF7FFF \\
VAL\_\_MAXD & \_DOUBLE & 1.701411834604923D$+$38 & FFFFFFFFFFFF7FFF \\
\hline
\end{tabular}
\end{center}
\end{quote}

Constants with names of the form \name{NUM\_\_MAX$<$T$>$} represent the
maximum (most positive) \emph{numbers} which can be represented on the machine.
The data types of these constants also match the quantities they describe, as
follows:

\begin{quote}
\begin{center}
\begin{tabular}{|l|l|r|r|}
\hline
\multicolumn{1}{|c|}{\textbf{Constant}} &
\multicolumn{1}{c|}{\textbf{Type}} &
\multicolumn{1}{c|}{\textbf{Value}} &
\multicolumn{1}{c|}{\textbf{Hexadecimal Pattern}} \\
\hline
NUM\_\_MAXUB & \_UBYTE & 255 & FF \\
NUM\_\_MAXB & \_BYTE & 127 & 7F \\
NUM\_\_MAXUW & \_UWORD & 65535 & FFFF \\
NUM\_\_MAXW & \_WORD & 32767 & 7FFF \\
NUM\_\_MAXI & \_INTEGER & 2147483647 & 7FFFFFFF \\
NUM\_\_MAXR & \_REAL & 1.7014117E$+$38 & FFFF7FFF \\
NUM\_\_MAXD & \_DOUBLE & 1.701411834604923D$+$38 & FFFFFFFFFFFF7FFF \\
\hline
\end{tabular}
\end{center}
\end{quote}

\subsection{Minimum Data Values}

Constants with names of the form \name{VAL\_\_MIN$<$T$>$} represent the
minimum (most negative) non-\emph{bad values} which can be represented
on the machine.
The data types of these constants match the quantities they describe, as
follows:

\begin{quote}
\begin{center}
\begin{tabular}{|l|l|r|r|}
\hline
\multicolumn{1}{|c|}{\textbf{Constant}} &
\multicolumn{1}{c|}{\textbf{Type}} &
\multicolumn{1}{c|}{\textbf{Value}} &
\multicolumn{1}{c|}{\textbf{Hexadecimal Pattern}} \\
\hline
VAL\_\_MINUB & \_UBYTE & 0 & 00 \\
VAL\_\_MINB & \_BYTE & $-$127 & 81 \\
VAL\_\_MINUW & \_UWORD & 0 & 0000 \\
VAL\_\_MINW & \_WORD & $-$32767 & 8001 \\
VAL\_\_MINI & \_INTEGER & $-$2147483647 & 80000001 \\
VAL\_\_MINR & \_REAL & $-$1.7014117E$+$38 & FFFEFFFF \\
VAL\_\_MIND & \_DOUBLE & $-$1.701411834604923D$+$38 & FFFEFFFFFFFFFFFF \\
\hline
\end{tabular}
\end{center}
\end{quote}

Constants with names of the form \name{NUM\_\_MIN$<$T$>$} represent the
minimum (most negative) \emph{numbers} which can be represented on the
machine.
The data types of these constants also match the quantities they describe, as
follows:

\begin{quote}
\begin{center}
\begin{tabular}{|l|l|r|r|}
\hline
\multicolumn{1}{|c|}{\textbf{Constant}} &
\multicolumn{1}{c|}{\textbf{Type}} &
\multicolumn{1}{c|}{\textbf{Value}} &
\multicolumn{1}{c|}{\textbf{Hexadecimal Pattern}} \\
\hline
NUM\_\_MINUB & \_UBYTE & 0 & 00 \\
NUM\_\_MINB & \_BYTE & $-$128 & 80 \\
NUM\_\_MINUW & \_UWORD & 0 & 0000 \\
NUM\_\_MINW & \_WORD & $-$32768 & 8000 \\
NUM\_\_MINI & \_INTEGER & $-$2147483648 & 80000000 \\
NUM\_\_MINR & \_REAL & $-$1.7014117E$+$38 & FFFFFFFF \\
NUM\_\_MIND & \_DOUBLE & $-$1.701411834604923D$+$38 & FFFFFFFFFFFFFFFF \\
\hline
\end{tabular}
\end{center}
\end{quote}

\subsection{Data Storage Requirements}

Constants with names of the form \name{VAL\_\_NB$<$T$>$} represent the
number of basic machine units (bytes) required to hold one value of each
data type.
All these constants are of type \name{\_INTEGER}, with values as follows:

\begin{quote}
\begin{center}
\begin{tabular}{|l|l|c|}
\hline
\multicolumn{1}{|c|}{\textbf{Constant}} &
\multicolumn{1}{c|}{\textbf{Type}} &
\textbf{Value} \\
\hline
VAL\_\_NBUB & \_INTEGER & 1 \\
VAL\_\_NBB & \_INTEGER & 1 \\
VAL\_\_NBUW & \_INTEGER & 2 \\
VAL\_\_NBW & \_INTEGER & 2 \\
VAL\_\_NBI & \_INTEGER & 4 \\
VAL\_\_NBR & \_INTEGER & 4 \\
VAL\_\_NBD & \_INTEGER & 8 \\
\hline
\end{tabular}
\end{center}
\end{quote}

\subsection{Smallest Value}

Constants with names of the form \name{VAL\_\_SML$<$T$>$} represent the
smallest positive (non-zero) value which can be represented on the machine.
The data types of these constants match the quantities they describe, as
follows:

\begin{quote}
\begin{center}
\begin{tabular}{|l|l|r|r|}
\hline
\multicolumn{1}{|c|}{\textbf{Constant}} &
\multicolumn{1}{c|}{\textbf{Type}} &
\multicolumn{1}{c|}{\textbf{Value}} &
\multicolumn{1}{c|}{\textbf{Hexadecimal Pattern}} \\
\hline
VAL\_\_SMLUB & \_UBYTE & 1 & 01 \\
VAL\_\_SMLB & \_BYTE & 1 & 01 \\
VAL\_\_SMLUW & \_UWORD & 1 & 0001 \\
VAL\_\_SMLW & \_WORD & 1 & 0001 \\
VAL\_\_SMLI & \_INTEGER & 1 & 00000001 \\
VAL\_\_SMLR & \_REAL & 2.9387359E$-$39 & 00000080 \\
VAL\_\_SMLD & \_DOUBLE & 2.9387358770557188E$-$39 & 0000000000000080 \\
\hline
\end{tabular}
\end{center}
\end{quote}

\subsection{Character String Sizes}

Constants with names of the form \name{VAL\_\_SZ$<$T$>$} specify the maximum
number of characters required to represent each numerical data type when
formatted as a decimal string.
All these constants are of type \name{\_INTEGER} and should be used when
declaring character variables to hold formatted numerical data.
Their values are:

\begin{quote}
\begin{center}
\begin{tabular}{|l|l|c|}
\hline
\multicolumn{1}{|c|}{\textbf{Constant}} &
\multicolumn{1}{c|}{\textbf{Type}} &
\textbf{Value} \\
\hline
VAL\_\_SZUB & \_INTEGER & 3 \\
VAL\_\_SZB & \_INTEGER & 4 \\
VAL\_\_SZUW & \_INTEGER & 5 \\
VAL\_\_SZW & \_INTEGER & 6 \\
VAL\_\_SZI & \_INTEGER & 11 \\
VAL\_\_SZR & \_INTEGER & 14 \\
VAL\_\_SZD & \_INTEGER & 22 \\
\hline
\end{tabular}
\end{center}
\end{quote}

\newpage
\section{COMPILING AND LINKING}

\subsection{Unix}
\label{compunix}
Before attempting to compile or link applications, the Starlink
executables directory (normally /star/bin) should be added to your PATH.
Links to the installed include files for PRIMDAT may then be set up
in your current working directory by executing the command:

\begin{terminalv}
% prm_dev
\end{terminalv}

and applications containing INCLUDE file names in upper case, in the normal way,
may be compiled and linked with the commands:

\begin{terminalv}
% f77 -o prog prog.f `prm_link`
\end{terminalv}

for stand-alone applications, or:

\begin{terminalv}
% alink prog.f `prm_link_adam`
\end{terminalv}

for ADAM applications.

\subsection{VMS}
The files required for compilation and linking with the routines in this
package reside in a directory with logical name \name{PRIMDAT\_DIR}.
Before attempting to compile or link applications, logical names must be
defined for the required files by executing the \name{DCL} command:

\begin{terminalv}
$ @PRIMDAT_DIR:START
\end{terminalv}

Applications which contain Fortran \name{INCLUDE} statements referring to
files in this directory may then be compiled.

Applications should be linked with routines from this package using the link
options file \name{PRM\_LINK}.
For instance, to link an application called \name{PROG}, the \name{DCL}
command:

\begin{terminalv}
$ LINK PROG,PRM_LINK/OPT
\end{terminalv}

might be used.
To link an \name{ADAM} application, the \name{\$LINK} command would be
replaced by the \name{ADAM} command \name{\$ALINK}.

\section{CHANGES IN THIS DOCUMENT}
Although PRIMDAT is now available for Unix platforms, the major part of this
document has not been revised and is still VMS oriented.
The use of PRIMDAT is similar for both systems but note that exception
handling is not available for all platforms (in particular, not for
Alpha OSF/1).

Section \ref{compunix} has been added for Unix users.

\appendix

\newpage

\section{PERFORMANCE STATISTICS}

\label{appendix:statistics}

This Appendix gives performance statistics for the \name{PRIMDAT} routines
in the form of typical execution times.
These are provided primarily to assist in the choice of routine for a
particular purpose, but they also provide a benchmark against which
alternative algorithms and future improvements can be judged.

In practice, execution times depend on many factors, such as the
particular processor being used, the number of page faults generated
and (in many cases) the routine argument values.
Consequently, great care should be exercised if comparing the figures given
here with the performance of a real application.
Nevertheless, the figures do give a good indication of the \textbf{relative}
efficiency of the routines, and are generally repeatable within 5 or 10 per
cent.
Users who know of more efficient algorithms are encouraged to contact the
author so that they may be tested under identical conditions and, if
appropriate, incorporated into this package.

The statistics presented here have been gathered on the Durham
\name{MicroVAX}~II (\name{DUMV1}), using a program which processes a
sequence of data obtained from an input array (or arrays) and returns the
results to a separate output array.
For instance, in the case of the \name{NUM\_ADDR} function, it is equivalent
to the simple loop:

\begin{terminalv}
  DO 1 I = 1, N
    A( I ) = NUM_ADDR( B( I ) , C( I ) )
1 CONTINUE
\end{terminalv}

The figures given represent the \name{CPU} time used to calculate one result
(\emph{i.e.} the total \name{CPU} time used by this loop divided by
\fortvar{N}), each being the median of five separate determinations.
Note that the cost of executing the loop and of accessing the arguments and
assigning the result is included, since these overheads will typically
feature in most real applications.

The cost of page faults is not included, however.
These will usually occur when data arrays are accessed for the first time
(and subsequently if the array is large or has not been accessed for a while).
The cost of page faults is difficult to quantify, as it may be
installation-dependent and will also depend on the level of system activity
at the time.
However, as a very rough guide, page faulting on the Durham \name{MicroVAX}~II
adds a broad average of around 7$\mu$s to the \name{CPU} time figure for
routines which access a single \name{\_REAL} argument array and around 11$\mu$s
for routines accessing two \name{\_REAL} argument arrays.
Page faults also impose a larger \textbf{elapsed} time overhead, so that their
effect can be substantial and can often make small apparent differences in
routine performance seem insignificant in practice.

\newpage

\begin{table}[h]
\begin{center}
\begin{scriptsize}
\begin{tabular}{|l|r|r||r|r||r|r||r|r||r|r||r|r||r|r|}
\cline{2-15}
\multicolumn{1}{c}{} & \multicolumn{14}{|c|}{\emph{Data Type Code $<$T$>$}}\\
\hline
\multicolumn{1}{|c}{\begin{tabular}{c}VAL\_ \\ \emph{Function} \end{tabular}} &
\multicolumn{2}{|c||}{\textbf{D}} &
\multicolumn{2}{c||}{\textbf{R}} &
\multicolumn{2}{c||}{\textbf{I}} &
\multicolumn{2}{c||}{\textbf{W}} &
\multicolumn{2}{c||}{\textbf{UW}} &
\multicolumn{2}{c||}{\textbf{B}} &
\multicolumn{2}{c|}{\textbf{UB}}\\
\hline
%   NOTE TO AUTHOR - the contents of the STATS_VAL.TEX file should be
%                    included at this point...
ADD$<$T$>$ &  100 &   82 &   94 &   78 &   89 &   76 &   89 &   77 &   95 &   83 &   89 &   76 &   97 &   85 \\
SUB$<$T$>$ &  100 &   82 &   94 &   79 &   89 &   76 &   89 &   77 &   95 &   83 &   88 &   76 &   97 &   85 \\
MUL$<$T$>$ &  102 &   83 &   95 &   79 &   93 &   81 &   94 &   82 &   99 &   88 &   92 &   79 &  101 &   91 \\
DIV$<$T$>$ &  105 &   85 &   96 &   81 &  110 &   98 &  111 &   99 &  109 &   97 &  105 &   93 &  111 &   98 \\
IDV$<$T$>$ &  151 &  134 &  139 &  122 &  103 &   91 &   98 &   85 &  103 &   90 &   94 &   82 &  105 &   94 \\
PWR$<$T$>$ &  410 &  395 &  287 &  270 &  140 &  127 &  139 &  127 &  140 &  128 &  135 &  124 &  140 &  129 \\
NEG$<$T$>$ &   51 &   44 &   48 &   42 &   47 &   48 &   47 &   48 &   59 &   57 &   48 &   48 &   52 &   51 \\
SQRT$<$T$>$ &  120 &  115 &  111 &  107 &  114 &  113 &  115 &  113 &  122 &  119 &  115 &  114 &  113 &  111 \\
LOG$<$T$>$ &  186 &  178 &  134 &  130 &  137 &  136 &  140 &  136 &  149 &  146 &  140 &  137 &  141 &  139 \\
LG10$<$T$>$ &  194 &  190 &  141 &  137 &  146 &  143 &  146 &  143 &  157 &  154 &  146 &  143 &  149 &  147 \\
EXP$<$T$>$ &  198 &  188 &  163 &  153 &  167 &  160 &  168 &  160 &  173 &  167 &  168 &  158 &  182 &  175 \\
ABS$<$T$>$ &   53 &   47 &   48 &   42 &   49 &   48 &   54 &   55 &   45 &   41 &   49 &   50 &   43 &   43 \\
NINT$<$T$>$ &  121 &  112 &  108 &  101 &   45 &   42 &   45 &   42 &   45 &   42 &   45 &   42 &   44 &   43 \\
INT$<$T$>$ &   93 &   87 &   82 &   79 &   45 &   41 &   45 &   42 &   45 &   42 &   45 &   41 &   43 &   43 \\
MAX$<$T$>$ &   68 &   58 &   66 &   59 &   58 &   53 &   56 &   52 &   64 &   60 &   60 &   56 &   66 &   62 \\
MIN$<$T$>$ &   68 &   57 &   66 &   59 &   56 &   52 &   57 &   53 &   65 &   61 &   61 &   56 &   66 &   62 \\
DIM$<$T$>$ &  103 &   83 &   95 &   80 &   97 &   85 &   88 &   76 &   65 &   61 &   95 &   82 &   66 &   63 \\
MOD$<$T$>$ &  242 &  222 &  183 &  165 &  108 &   99 &  112 &  103 &  114 &  101 &  112 &  100 &  117 &  106 \\
SIGN$<$T$>$ &  115 &   95 &  108 &   93 &  137 &  124 &  132 &  120 &   55 &   42 &  139 &  125 &   53 &   44 \\
SIN$<$T$>$ &  258 &  247 &  196 &  186 &      &      &      &      &      &      &      &      &      &      \\
SIND$<$T$>$ &  222 &  210 &  172 &  163 &      &      &      &      &      &      &      &      &      &      \\
COS$<$T$>$ &  198 &  194 &  151 &  147 &      &      &      &      &      &      &      &      &      &      \\
COSD$<$T$>$ &  245 &  232 &  202 &  192 &      &      &      &      &      &      &      &      &      &      \\
TAN$<$T$>$ &  370 &  360 &  273 &  265 &      &      &      &      &      &      &      &      &      &      \\
TAND$<$T$>$ &  343 &  333 &  265 &  258 &      &      &      &      &      &      &      &      &      &      \\
ASIN$<$T$>$ &  255 &  248 &  206 &  202 &      &      &      &      &      &      &      &      &      &      \\
ASND$<$T$>$ &  258 &  253 &  206 &  202 &      &      &      &      &      &      &      &      &      &      \\
ACOS$<$T$>$ &  270 &  265 &  216 &  210 &      &      &      &      &      &      &      &      &      &      \\
ACSD$<$T$>$ &  273 &  270 &  218 &  212 &      &      &      &      &      &      &      &      &      &      \\
ATAN$<$T$>$ &  167 &  162 &  125 &  121 &      &      &      &      &      &      &      &      &      &      \\
ATND$<$T$>$ &  170 &  163 &  125 &  120 &      &      &      &      &      &      &      &      &      &      \\
ATN2$<$T$>$ &  232 &  218 &  194 &  180 &      &      &      &      &      &      &      &      &      &      \\
AT2D$<$T$>$ &  235 &  222 &  192 &  178 &      &      &      &      &      &      &      &      &      &      \\
SINH$<$T$>$ &  196 &  182 &  167 &  155 &      &      &      &      &      &      &      &      &      &      \\
COSH$<$T$>$ &  255 &  238 &  212 &  198 &      &      &      &      &      &      &      &      &      &      \\
TANH$<$T$>$ &  247 &  237 &  170 &  163 &      &      &      &      &      &      &      &      &      &      \\
%   NOTE - end of file inclusion.
\hline
\multicolumn{1}{c}{} &
\multicolumn{1}{|c}{\textbf{T}} & \multicolumn{1}{|c||}{\textbf{F}} &
\multicolumn{1}{c}{\textbf{T}} & \multicolumn{1}{|c||}{\textbf{F}} &
\multicolumn{1}{c}{\textbf{T}} & \multicolumn{1}{|c||}{\textbf{F}} &
\multicolumn{1}{c}{\textbf{T}} & \multicolumn{1}{|c||}{\textbf{F}} &
\multicolumn{1}{c}{\textbf{T}} & \multicolumn{1}{|c||}{\textbf{F}} &
\multicolumn{1}{c}{\textbf{T}} & \multicolumn{1}{|c||}{\textbf{F}} &
\multicolumn{1}{c}{\textbf{T}} & \multicolumn{1}{|c|}{\textbf{F}}\\
\cline{2-15}
\multicolumn{1}{c}{} &
\multicolumn{14}{|c|}{\emph{BAD argument set?}}\\
\cline{2-15}
\end{tabular}
\end{scriptsize}
\caption{Approximate median execution times ($\mu$s per operation) for the
\name{VAL\_} arithmetic and mathematical functions.}
\label{table:valstats}
\end{center}
\end{table}

\newpage

\begin{table}[h]
\begin{center}
\begin{scriptsize}
\begin{tabular}{|l|r|r||r|r||r|r||r|r||r|r||r|r||r|r|}
\cline{2-15}
\multicolumn{1}{c}{} & \multicolumn{14}{|c|}{\emph{Data Type Code $<$T$>$}}\\
\hline
\multicolumn{1}{|c}{\begin{tabular}{c}VEC\_ \\ \emph{Routine} \end{tabular}} &
\multicolumn{2}{|c||}{\textbf{D}} &
\multicolumn{2}{c||}{\textbf{R}} &
\multicolumn{2}{c||}{\textbf{I}} &
\multicolumn{2}{c||}{\textbf{W}} &
\multicolumn{2}{c||}{\textbf{UW}} &
\multicolumn{2}{c||}{\textbf{B}} &
\multicolumn{2}{c|}{\textbf{UB}}\\
\hline
%   NOTE TO AUTHOR - the contents of the STATS_VEC.TEX file should be
%                    included at this point...
ADD$<$T$>$ &   21 & 10.6 & 16.6 &  8.6 & 11.2 &  6.6 & 11.7 &  6.3 & 17.0 & 12.8 & 11.4 &  6.8 & 19.4 & 14.8 \\
SUB$<$T$>$ &   21 & 10.6 & 16.6 &  8.6 & 11.2 &  6.6 & 11.7 &  6.3 & 17.1 & 12.8 & 11.5 &  6.8 & 19.3 & 14.8 \\
MUL$<$T$>$ &   23 & 12.2 & 17.4 &  9.3 & 15.2 & 10.6 & 16.5 & 11.1 &   21 & 17.0 & 15.1 & 10.1 &   25 &   20 \\
DIV$<$T$>$ &   25 & 14.5 & 18.4 & 10.4 &   25 &   21 &   26 &   22 &   32 &   28 &   26 &   22 &   32 &   28 \\
IDV$<$T$>$ &   77 &   68 &   66 &   56 & 18.7 & 13.9 &   20 & 14.9 &   24 &   20 & 16.7 & 12.1 &   28 &   23 \\
PWR$<$T$>$ &  343 &  337 &  224 &  208 &   66 &   60 &   74 &   63 &   70 &   65 &   66 &   63 &   69 &   69 \\
NEG$<$T$>$ & 10.7 &  6.1 &  8.3 &  4.8 &  8.1 &  7.5 &  8.1 &  7.5 & 13.0 & 10.8 &  8.1 &  7.4 & 13.8 & 11.8 \\
SQRT$<$T$>$ &   78 &   72 &   72 &   67 &   78 &   73 &   77 &   74 &   76 &   74 &   80 &   74 &   76 &   77 \\
LOG$<$T$>$ &  143 &  137 &   93 &   90 &  100 &   95 &   98 &   96 &  100 &   97 &  101 &   96 &  101 &   98 \\
LG10$<$T$>$ &  155 &  148 &  100 &   97 &  108 &  102 &  107 &  104 &  107 &  104 &  108 &  103 &  108 &  106 \\
EXP$<$T$>$ &  130 &  125 &   96 &   93 &  101 &  100 &  102 &  101 &  108 &  106 &  102 &  101 &  108 &  107 \\
ABS$<$T$>$ & 11.0 &  6.5 &  7.7 &  4.4 &  9.2 &  8.6 &  9.1 &  8.6 &  5.4 &  3.7 & 11.0 & 10.4 &  5.3 &  3.8 \\
NINT$<$T$>$ &   80 &   74 &   72 &   63 &  5.4 &  3.8 &  5.4 &  3.8 &  5.4 &  3.8 &  5.4 &  3.8 &  5.4 &  3.7 \\
INT$<$T$>$ &   53 &   48 &   47 &   43 &  5.4 &  3.8 &  5.4 &  3.8 &  5.4 &  3.8 &  5.4 &  3.8 &  5.4 &  3.8 \\
MAX$<$T$>$ &   20 & 11.8 & 14.3 &  8.3 &  9.0 &  6.2 &  8.9 &  6.3 & 15.1 & 12.6 & 11.1 &  8.5 & 17.0 & 14.4 \\
MIN$<$T$>$ &   20 & 11.6 & 14.9 &  8.5 &  9.6 &  6.5 &  9.5 &  6.5 & 15.4 & 13.4 & 11.3 &  9.1 & 17.6 & 14.8 \\
DIM$<$T$>$ &   22 & 13.8 & 16.8 & 10.6 & 12.3 &  9.3 & 12.8 &  8.8 & 15.7 & 13.2 & 13.6 & 12.1 & 17.8 & 15.0 \\
MOD$<$T$>$ &  185 &  172 &  120 &  110 &   27 &   24 &   28 &   26 &   33 &   29 &   31 &   29 &   38 &   35 \\
SIGN$<$T$>$ &   77 &   66 &   72 &   55 &   71 &   66 &   60 &   56 &  7.5 &  3.8 &   73 &   69 &  7.5 &  3.8 \\
SIN$<$T$>$ &  182 &  178 &  122 &  126 &      &      &      &      &      &      &      &      &      &      \\
SIND$<$T$>$ &  147 &  143 &  101 &  103 &      &      &      &      &      &      &      &      &      &      \\
COS$<$T$>$ &  158 &  155 &  112 &  109 &      &      &      &      &      &      &      &      &      &      \\
COSD$<$T$>$ &  170 &  165 &  130 &  133 &      &      &      &      &      &      &      &      &      &      \\
TAN$<$T$>$ &  297 &  293 &  200 &  202 &      &      &      &      &      &      &      &      &      &      \\
TAND$<$T$>$ &  267 &  263 &  192 &  198 &      &      &      &      &      &      &      &      &      &      \\
ASIN$<$T$>$ &  212 &  206 &  165 &  165 &      &      &      &      &      &      &      &      &      &      \\
ASND$<$T$>$ &  213 &  210 &  165 &  165 &      &      &      &      &      &      &      &      &      &      \\
ACOS$<$T$>$ &  225 &  222 &  173 &  172 &      &      &      &      &      &      &      &      &      &      \\
ACSD$<$T$>$ &  228 &  225 &  175 &  175 &      &      &      &      &      &      &      &      &      &      \\
ATAN$<$T$>$ &  124 &  121 &   85 &   82 &      &      &      &      &      &      &      &      &      &      \\
ATND$<$T$>$ &  129 &  124 &   85 &   82 &      &      &      &      &      &      &      &      &      &      \\
ATN2$<$T$>$ &  192 &  178 &  150 &  134 &      &      &      &      &      &      &      &      &      &      \\
AT2D$<$T$>$ &  194 &  180 &  149 &  134 &      &      &      &      &      &      &      &      &      &      \\
SINH$<$T$>$ &  126 &  129 &  104 &  105 &      &      &      &      &      &      &      &      &      &      \\
COSH$<$T$>$ &  190 &  183 &  149 &  140 &      &      &      &      &      &      &      &      &      &      \\
TANH$<$T$>$ &  206 &  194 &  137 &  137 &      &      &      &      &      &      &      &      &      &      \\
%   NOTE - end of file inclusion.
\hline
\multicolumn{1}{c}{} &
\multicolumn{1}{|c}{\textbf{T}} & \multicolumn{1}{|c||}{\textbf{F}} &
\multicolumn{1}{c}{\textbf{T}} & \multicolumn{1}{|c||}{\textbf{F}} &
\multicolumn{1}{c}{\textbf{T}} & \multicolumn{1}{|c||}{\textbf{F}} &
\multicolumn{1}{c}{\textbf{T}} & \multicolumn{1}{|c||}{\textbf{F}} &
\multicolumn{1}{c}{\textbf{T}} & \multicolumn{1}{|c||}{\textbf{F}} &
\multicolumn{1}{c}{\textbf{T}} & \multicolumn{1}{|c||}{\textbf{F}} &
\multicolumn{1}{c}{\textbf{T}} & \multicolumn{1}{|c|}{\textbf{F}}\\
\cline{2-15}
\multicolumn{1}{c}{} &
\multicolumn{14}{|c|}{\emph{BAD argument set?}}\\
\cline{2-15}
\end{tabular}
\end{scriptsize}
\caption{Approximate median execution times ($\mu$s per operation) for the
\name{VEC\_} arithmetic and mathematical routines.}
\label{table:vecstats}
\end{center}
\end{table}

\newpage

\begin{table}[h]
\begin{center}
\begin{footnotesize}
\begin{tabular}{|l|r|r|r|r|r|r|r|}
\cline{2-8}
\multicolumn{1}{c}{} & \multicolumn{7}{|c|}{\emph{Data Type Code $<$T$>$}}\\
\hline
\multicolumn{1}{|c}{\begin{tabular}{c}NUM\_ \\ \emph{Function} \end{tabular}} &
\multicolumn{1}{|c|}{\textbf{D}} &
\multicolumn{1}{|c|}{\textbf{R}} &
\multicolumn{1}{|c|}{\textbf{I}} &
\multicolumn{1}{|c|}{\textbf{W}} &
\multicolumn{1}{|c|}{\textbf{UW}} &
\multicolumn{1}{|c|}{\textbf{B}} &
\multicolumn{1}{|c|}{\textbf{UB}}\\
\hline
%   NOTE TO AUTHOR - the contents of the STATS_NUM.TEX file should be
%                    included at this point...
ADD$<$T$>$ &  8.5 &  6.4 &  4.5 &  4.4 & 10.6 &  4.0 & 12.1 \\
SUB$<$T$>$ &  8.4 &  6.5 &  4.5 &  4.4 & 10.6 &  3.9 & 12.1 \\
MUL$<$T$>$ & 10.3 &  7.2 &  8.6 &  9.2 & 14.8 &  7.4 & 17.2 \\
DIV$<$T$>$ & 12.6 &  8.4 & 18.6 & 19.2 &   25 & 18.8 &   25 \\
IDV$<$T$>$ &   65 &   55 & 11.5 & 13.2 & 17.8 &  9.2 &   21 \\
PWR$<$T$>$ &  335 &  213 &   60 &   62 &   66 &   60 &   67 \\
NEG$<$T$>$ &  5.2 &  4.4 &  3.5 &  3.5 &  9.4 &  3.1 & 10.1 \\
SQRT$<$T$>$ &   73 &   68 &   73 &   74 &   75 &   72 &   73 \\
LOG$<$T$>$ &  143 &   90 &   96 &   97 &   96 &   94 &   95 \\
LG10$<$T$>$ &  154 &   97 &  102 &  103 &  104 &  102 &  103 \\
EXP$<$T$>$ &  122 &   92 &   97 &   98 &  102 &   93 &  100 \\
ABS$<$T$>$ &  6.6 &  4.2 &  4.4 &  4.4 &  0.6 &  5.6 &  0.3 \\
NINT$<$T$>$ &   86 &   72 &  1.1 &  0.5 &  0.5 &  0.3 &  0.3 \\
INT$<$T$>$ &   51 &   44 &  1.1 &  0.5 &  0.5 &  0.3 &  0.3 \\
MAX$<$T$>$ & 11.2 &  8.1 &  6.3 &  6.1 & 11.6 &  7.0 & 13.3 \\
MIN$<$T$>$ & 10.5 &  8.1 &  6.3 &  6.3 & 12.1 &  7.2 & 13.7 \\
DIM$<$T$>$ & 10.4 &  7.9 &  6.4 &  6.1 & 12.4 &  7.5 & 13.9 \\
MOD$<$T$>$ &  178 &  112 &   20 &   22 &   27 &   25 &   31 \\
SIGN$<$T$>$ &   74 &   72 &   73 &   69 & 18.4 &   69 & 18.4 \\
SIN$<$T$>$ &  176 &  118 &      &      &      &      &      \\
SIND$<$T$>$ &  140 &   95 &      &      &      &      &      \\
COS$<$T$>$ &  154 &  110 &      &      &      &      &      \\
COSD$<$T$>$ &  162 &  124 &      &      &      &      &      \\
TAN$<$T$>$ &  291 &  195 &      &      &      &      &      \\
TAND$<$T$>$ &  264 &  186 &      &      &      &      &      \\
ASIN$<$T$>$ &  198 &  158 &      &      &      &      &      \\
ASND$<$T$>$ &  201 &  157 &      &      &      &      &      \\
ACOS$<$T$>$ &  214 &  166 &      &      &      &      &      \\
ACSD$<$T$>$ &  218 &  167 &      &      &      &      &      \\
ATAN$<$T$>$ &  120 &   84 &      &      &      &      &      \\
ATND$<$T$>$ &  124 &   84 &      &      &      &      &      \\
ATN2$<$T$>$ &  184 &  142 &      &      &      &      &      \\
AT2D$<$T$>$ &  185 &  140 &      &      &      &      &      \\
SINH$<$T$>$ &  127 &   98 &      &      &      &      &      \\
COSH$<$T$>$ &  180 &  140 &      &      &      &      &      \\
TANH$<$T$>$ &  210 &  130 &      &      &      &      &      \\
%   NOTE - end of file inclusion.
\hline
\end{tabular}
\end{footnotesize}
\caption{Approximate median execution times ($\mu$s per operation) for the
\name{NUM\_} arithmetic and mathematical functions.}
\label{table:numstats}
\end{center}
\end{table}

\newpage

\begin{table}[h]
\begin{center}
\begin{small}
\begin{tabular}{|l|r|r||r|r||r|r||r|r||r|r||r|r||r|r|}
\cline{2-15}
\multicolumn{1}{c}{} & \multicolumn{14}{|c|}{\emph{Result Type Code $<$T$>$}}\\
\hline
\multicolumn{1}{|c}{\begin{tabular}{c}VAL\_ \\ \emph{Function} \end{tabular}} &
\multicolumn{2}{|c||}{\textbf{D}} &
\multicolumn{2}{c||}{\textbf{R}} &
\multicolumn{2}{c||}{\textbf{I}} &
\multicolumn{2}{c||}{\textbf{W}} &
\multicolumn{2}{c||}{\textbf{UW}} &
\multicolumn{2}{c||}{\textbf{B}} &
\multicolumn{2}{c|}{\textbf{UB}}\\
\hline
%   NOTE TO AUTHOR - The contents of the file STATS_VAL_CVT.TEX should be
%                    included at this point...
DTO$<$T$>$ &   35 &   45 &   83 &   72 &   86 &   73 &   87 &   75 &   92 &   79 &   88 &   75 &   97 &   86 \\
RTO$<$T$>$ &   50 &   49 &   33 &   33 &   84 &   74 &   84 &   74 &   93 &   86 &   84 &   74 &   93 &   83 \\
ITO$<$T$>$ &   49 &   51 &   49 &   44 &   33 &   33 &   80 &   71 &   87 &   79 &   79 &   72 &   87 &   79 \\
WTO$<$T$>$ &   49 &   51 &   48 &   44 &   45 &   42 &   33 &   33 &   90 &   84 &   80 &   71 &   86 &   77 \\
UWTO$<$T$>$ &   50 &   51 &   48 &   45 &   45 &   41 &   84 &   79 &   33 &   33 &   79 &   72 &   87 &   79 \\
BTO$<$T$>$ &   50 &   46 &   48 &   44 &   45 &   42 &   45 &   42 &   89 &   83 &   32 &   33 &   87 &   80 \\
UBTO$<$T$>$ &   50 &   46 &   48 &   45 &   45 &   41 &   46 &   42 &   45 &   41 &   81 &   72 &   26 &   26 \\
%   NOTE - end of file inclusion.
\hline
\multicolumn{1}{c}{} &
\multicolumn{1}{|c}{\textbf{T}} & \multicolumn{1}{|c||}{\textbf{F}} &
\multicolumn{1}{c}{\textbf{T}} & \multicolumn{1}{|c||}{\textbf{F}} &
\multicolumn{1}{c}{\textbf{T}} & \multicolumn{1}{|c||}{\textbf{F}} &
\multicolumn{1}{c}{\textbf{T}} & \multicolumn{1}{|c||}{\textbf{F}} &
\multicolumn{1}{c}{\textbf{T}} & \multicolumn{1}{|c||}{\textbf{F}} &
\multicolumn{1}{c}{\textbf{T}} & \multicolumn{1}{|c||}{\textbf{F}} &
\multicolumn{1}{c}{\textbf{T}} & \multicolumn{1}{|c|}{\textbf{F}}\\
\cline{2-15}
\multicolumn{1}{c}{} &
\multicolumn{14}{|c|}{\emph{BAD argument set?}}\\
\cline{2-15}
\end{tabular}
\end{small}
\caption{Approximate median execution times ($\mu$s per operation) for the
\name{VAL\_} type conversion functions.}
\label{table:valcvtstats}
\end{center}
\end{table}

\begin{table}[h]
\begin{center}
\begin{small}
\begin{tabular}{|l|r|r||r|r||r|r||r|r||r|r||r|r||r|r|}
\cline{2-15}
\multicolumn{1}{c}{} & \multicolumn{14}{|c|}{\emph{Result Type Code $<$T$>$}}\\
\hline
\multicolumn{1}{|c}{\begin{tabular}{c}VEC\_ \\ \emph{Routine} \end{tabular}} &
\multicolumn{2}{|c||}{\textbf{D}} &
\multicolumn{2}{c||}{\textbf{R}} &
\multicolumn{2}{c||}{\textbf{I}} &
\multicolumn{2}{c||}{\textbf{W}} &
\multicolumn{2}{c||}{\textbf{UW}} &
\multicolumn{2}{c||}{\textbf{B}} &
\multicolumn{2}{c|}{\textbf{UB}}\\
\hline
%   NOTE TO AUTHOR - The contents of the file STATS_VEC_CVT.TEX should be
DTO$<$T$>$ &  2.4 &  2.2 & 12.6 &  8.0 & 14.7 &  9.6 & 15.6 &  9.8 &   21 & 15.8 & 15.6 & 10.0 &   21 & 16.4 \\
RTO$<$T$>$ &  9.1 &  5.2 &  1.2 &  1.2 & 12.4 &  8.4 & 13.3 &  9.3 & 18.5 & 14.1 & 13.4 &  9.2 & 18.8 & 15.6 \\
ITO$<$T$>$ &  9.0 &  6.8 &  8.5 &  6.3 &  1.2 &  1.2 &  8.2 &  6.8 & 13.4 & 11.6 &  8.2 &  6.5 & 13.9 & 12.0 \\
WTO$<$T$>$ &  9.1 &  6.8 &  8.4 &  6.3 &  5.8 &  3.6 &  0.6 &  0.7 & 14.2 & 11.8 &  8.0 &  6.5 & 13.4 & 11.1 \\
UWTO$<$T$>$ &  9.5 &  7.3 &  9.0 &  6.6 &  5.4 &  3.2 &  8.5 &  6.6 &  0.7 &  0.6 &  8.9 &  6.3 & 14.3 & 12.6 \\
BTO$<$T$>$ &  9.1 &  6.8 &  8.5 &  6.1 &  5.8 &  3.6 &  5.8 &  3.5 & 14.2 & 11.6 &  0.4 &  0.4 & 13.6 & 11.8 \\
UBTO$<$T$>$ &  9.5 &  7.3 &  9.0 &  6.7 &  5.4 &  3.2 &  6.5 &  4.2 &  5.3 &  3.2 &  8.9 &  6.3 &  0.4 &  0.4 \\
%   NOTE - end of file inclusion.
\hline
\multicolumn{1}{c}{} &
\multicolumn{1}{|c}{\textbf{T}} & \multicolumn{1}{|c||}{\textbf{F}} &
\multicolumn{1}{c}{\textbf{T}} & \multicolumn{1}{|c||}{\textbf{F}} &
\multicolumn{1}{c}{\textbf{T}} & \multicolumn{1}{|c||}{\textbf{F}} &
\multicolumn{1}{c}{\textbf{T}} & \multicolumn{1}{|c||}{\textbf{F}} &
\multicolumn{1}{c}{\textbf{T}} & \multicolumn{1}{|c||}{\textbf{F}} &
\multicolumn{1}{c}{\textbf{T}} & \multicolumn{1}{|c||}{\textbf{F}} &
\multicolumn{1}{c}{\textbf{T}} & \multicolumn{1}{|c|}{\textbf{F}}\\
\cline{2-15}
\multicolumn{1}{c}{} &
\multicolumn{14}{|c|}{\emph{BAD argument set?}}\\
\cline{2-15}
\end{tabular}
\end{small}
\caption{Approximate median execution times ($\mu$s per operation) for the
\name{VEC\_} type conversion routines.}
\label{table:veccvtstats}
\end{center}
\end{table}

\begin{table}[h]
\begin{center}
\begin{small}
\begin{tabular}{|l|r|r|r|r|r|r|r|}
\cline{2-8}
\multicolumn{1}{c}{} & \multicolumn{7}{|c|}{\emph{Result Type Code $<$T$>$}}\\
\hline
\multicolumn{1}{|c}{\begin{tabular}{c}NUM\_ \\ \emph{Function} \end{tabular}} &
\multicolumn{1}{|c|}{\textbf{D}} &
\multicolumn{1}{|c|}{\textbf{R}} &
\multicolumn{1}{|c|}{\textbf{I}} &
\multicolumn{1}{|c|}{\textbf{W}} &
\multicolumn{1}{|c|}{\textbf{UW}} &
\multicolumn{1}{|c|}{\textbf{B}} &
\multicolumn{1}{|c|}{\textbf{UB}}\\
\hline
%   NOTE TO AUTHOR - the contents of the STATS_NUM_CVT.TEX file should be
%                    included at this point...
DTO$<$T$>$ &  2.2 &  5.4 &  6.6 &  7.6 & 12.6 &  7.6 & 13.3 \\
RTO$<$T$>$ &  5.1 &  1.1 &  6.2 &  6.9 & 12.0 &  7.0 & 12.7 \\
ITO$<$T$>$ &  6.9 &  6.4 &  1.1 &  3.7 &  9.1 &  3.7 &  9.4 \\
WTO$<$T$>$ &  6.7 &  6.2 &  3.5 &  0.5 &  9.7 &  3.7 &  8.8 \\
UWTO$<$T$>$ &  7.1 &  6.7 &  3.1 &  4.0 &  0.5 &  4.1 &  9.9 \\
BTO$<$T$>$ &  6.9 &  6.1 &  3.5 &  3.5 &  9.4 &  0.3 &  9.4 \\
UBTO$<$T$>$ &  7.1 &  6.7 &  3.1 &  4.1 &  3.1 &  3.9 &  0.3 \\
%   NOTE - end of file inclusion.
\hline
\end{tabular}
\end{small}
\caption{Approximate median execution times ($\mu$s per operation) for the
\name{NUM\_} type conversion functions.}
\label{table:numcvtstats}
\end{center}
\end{table}

\clearpage

\section{ERROR CODES \& MESSAGES}

\label{appendix:errors}

This Appendix describes the error conditions recognised by this package
and the associated codes and messages.
If it is necessary to test for any of these error conditions, then symbolic
constants (defined by Fortran \name{PARAMETER} statements) should be used to
identify the error code in question.

The names of \name{PRIMDAT} error codes are of the form:

\begin{terminalv}
PRM__<code>
\end{terminalv}

where $<$code$>$ specifies the error condition.
Their numerical values are defined in the file \name{PRM\_ERR}, and may be
incorporated into an application with the statement:

\begin{terminalv}
INCLUDE 'PRM_ERR'
\end{terminalv}

The following gives the names of the symbolic constants defined in this
file, together with the associated error messages and an explanation of each
error condition:

\begin{description}

\item \textbf{PRM\_\_ARGIN, function argument invalid}\\
An invalid argument was supplied to a mathematical function.

\item \textbf{PRM\_\_FLTDZ, floating point divide by zero}\\
An attempt was made to perform floating point division by zero.

\item \textbf{PRM\_\_FLTOF, floating point overflow}\\
The absolute value of the result of a floating point calculation or type
conversion exceeded the largest floating point number which can be
represented on the machine.

\item \textbf{PRM\_\_FLTUF, floating point underflow}\\
The absolute value of the result of a floating point calculation or type
conversion was smaller than the smallest non-zero floating point number
which can be represented on the machine.
This error condition will not normally be used on a \name{VAX} unless
underflow detection has been explicitly enabled.

\item \textbf{PRM\_\_INTDZ, integer divide by zero}\\
An attempt was made to perform integer division by zero.

\item \textbf{PRM\_\_INTOF, integer overflow}\\
The result of an integer (non-floating point) calculation or type conversion
was outside the number range which can be represented by the data type in
question.

\item \textbf{PRM\_\_LOGZN, logarithm of zero or negative number}\\
An attempt was made to take the logarithm of zero or a negative number.

\item \textbf{PRM\_\_SLOST, numerical significance lost}\\
Numerical significance was lost during evaluation of a mathematical
function; a result could not be calculated with any meaningful precision.

\item \textbf{PRM\_\_SQRNG, square root of negative number}\\
An attempt was made to take the square root of a negative number.

\item \textbf{PRM\_\_UDEXP, undefined exponentiation}\\
An attempt was made to raise the number zero to a negative power, or to raise a
negative number to a non-integer power.

\end{description}

\end{document}
